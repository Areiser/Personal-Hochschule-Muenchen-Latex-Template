\section{Einleitung}

Im Rahmen des Hauptseminars für Embedded Computing, einem Masterstudiengang im Fach Informatik der Hochschule München, wurde von fünf Studierenden ein universelles Kommunikationsframework für Smart Objects entwickelt.


\subsection{Anwendungsbereich Parkleitsystem}

Bestehende Systeme zeigen die Anzahl der freien Plätze auf Schilder an und Fahrer versuchen dann den entsprechenden Parkplatz oder das Parkhaus zu finden. Möchte der Fahrer in einer fremden Stadt mit einem Navigationsgerät einen Parkplatz finden, muss er nachdem er das Parkleitsystem gesehen hat den entsprechenden Parkplatz erst in sein Gerät eingeben [3].

Im Rahmen des Hauptseminars für Embedded Computing, einem Masterstudiengang im Fach Informatik der Hochschule München, wurde von fünf Studierenden ein universelles Kommunikationsframework für Smart Objects entwickelt.


\subsection{Anwendungsbereich Parkleitsystem}

Bestehende Systeme zeigen die Anzahl der freien Plätze auf Schilder an und Fahrer versuchen dann den entsprechenden Parkplatz oder das Parkhaus zu finden. Möchte der Fahrer in einer fremden Stadt mit einem Navigationsgerät einen Parkplatz finden, muss er nachdem er das Parkleitsystem gesehen hat den entsprechenden Parkplatz erst in sein Gerät eingeben [3].

Im Rahmen des Hauptseminars für Embedded Computing, einem Masterstudiengang im Fach Informatik der Hochschule München, wurde von fünf Studierenden ein universelles Kommunikationsframework für Smart Objects entwickelt.


\subsection{Anwendungsbereich Parkleitsystem}

Bestehende Systeme zeigen die Anzahl der freien Plätze auf Schilder an und Fahrer versuchen dann den entsprechenden Parkplatz oder das Parkhaus zu finden. Möchte der Fahrer in einer fremden Stadt mit einem Navigationsgerät einen Parkplatz finden, muss er nachdem er das Parkleitsystem gesehen hat den entsprechenden Parkplatz erst in sein Gerät eingeben [3].
