\documentclass[bibnumbered,titlepage]{scrartcl}

\usepackage{config}

\newcommand{\utsection}[2]{%Section mit Untertitel (ut)
    \section[#1]{#1\newline\normalfont\small\textit{Von: #2}}
}
\newcommand{\utsubsection}[2]{%Subsection mit Untertitel (ut)
    \subsection[#1]{#1\newline\normalfont\small\textit{Von: #2}}
}

\title{Wie lautet der Titel vom nächsten Kapitel?}
\subject{Masterarbeit}
\author{Thomas Waldecker, B.Eng.}
\date{}
\newcommand{\abgabe}{März 2013}
\newcommand{\betreuer}{Prof. Dr. No \textit{Hochschule München}}
\newcommand{\betreuera}{Dipl. Ing. XX \textit{Engi GmbH}}


\newacronym{bts}{BTS}{Base Transceiver Station}
\newacronym{bsc}{BSC}{Base Station Controller}
\newacronym{msc}{MSC}{Mobile Switching Center}
\newacronym{sacch}{SACCH}{Slow Associated Control Channel}
\newacronym{sabm}{SABM}{Set Asynchronous Balanced Mode}
\newacronym{UA}{UA}{Unnumbered Acknowledgement}
\newacronym{ARFCN}{ARFCN}{Absolute Radio Frequency Channel Number}
\newacronym{BCCH}{BCCH}{Broadcast Control CHannel}
\newacronym{TCH}{TCH}{Traffic CHannel}
\newacronym{TCH/F}{TCH/F}{full rate TCH}

%% jetzt geht's los
\begin{document}

\maketitle
\begin{abstract}
\section*{Abstract}
Diese Seminararbeit entstand im Hauptseminar Embedded Systems und ist Teil des gesamten Projekts, ein Framework für die Vernetzung von eingebetteten Systemen. Dieses Framework soll als Grundlage zur Vernetzung von allen möglichen zukünftigen eingebetteten Systemen dienen. In der Einleitung werden dazu zwei Anwendugsfälle, ein Parkhaus, das die Zustände aller einzelnen Parkplätze kennt, und ein Logistikcontainer, der programmiert werden kann um verschiedene Daten wie z.B. die Position oder die Temperatur der Ware aufzunehmen. Im Inhalt dieser Seminararbeit werden Charakteristiken erläutert die für typische Sensorknoten in drahtlosen Netzwerken gelten. Diese müssen von Betriebssystemarchitekturen berücksichtigt werden. Die zwei aktuellsten und am weitesten fortgeschrittensten Sensornetzbetriebssysteme TinyOS und Contiki OS werden vorgestellt und miteinander verglichen. Am Ende wird eine proof-of-concept Implementierung eines intelligenten Parkhauses mit Contiki OS gezeigt.
\end{abstract}

\tableofcontents

\section{Einleitung}

Im Rahmen des Hauptseminars für Embedded Computing, einem Masterstudiengang im Fach Informatik der Hochschule München, wurde von fünf Studierenden ein universelles Kommunikationsframework für Smart Objects entwickelt.


\subsection{Anwendungsbereich Parkleitsystem}

Bestehende Systeme zeigen die Anzahl der freien Plätze auf Schilder an und Fahrer versuchen dann den entsprechenden Parkplatz oder das Parkhaus zu finden. Möchte der Fahrer in einer fremden Stadt mit einem Navigationsgerät einen Parkplatz finden, muss er nachdem er das Parkleitsystem gesehen hat den entsprechenden Parkplatz erst in sein Gerät eingeben [3].

Im Rahmen des Hauptseminars für Embedded Computing, einem Masterstudiengang im Fach Informatik der Hochschule München, wurde von fünf Studierenden ein universelles Kommunikationsframework für Smart Objects entwickelt.


\subsection{Anwendungsbereich Parkleitsystem}

Bestehende Systeme zeigen die Anzahl der freien Plätze auf Schilder an und Fahrer versuchen dann den entsprechenden Parkplatz oder das Parkhaus zu finden. Möchte der Fahrer in einer fremden Stadt mit einem Navigationsgerät einen Parkplatz finden, muss er nachdem er das Parkleitsystem gesehen hat den entsprechenden Parkplatz erst in sein Gerät eingeben [3].

Im Rahmen des Hauptseminars für Embedded Computing, einem Masterstudiengang im Fach Informatik der Hochschule München, wurde von fünf Studierenden ein universelles Kommunikationsframework für Smart Objects entwickelt.


\subsection{Anwendungsbereich Parkleitsystem}

Bestehende Systeme zeigen die Anzahl der freien Plätze auf Schilder an und Fahrer versuchen dann den entsprechenden Parkplatz oder das Parkhaus zu finden. Möchte der Fahrer in einer fremden Stadt mit einem Navigationsgerät einen Parkplatz finden, muss er nachdem er das Parkleitsystem gesehen hat den entsprechenden Parkplatz erst in sein Gerät eingeben [3].



\appendix

\glsaddall % alle definierten abk�rzungen zeigen
\printglossary[numberedsection, title=Glossar] %Glossar, section mit nummer.

\bibliography{verzeichnis}
\bibliographystyle{alpha}

\end{document}
